\documentclass[a4paper]{jarticle}
\usepackage{tani_resume}
\usepackage{epstopdf}
\usepackage{graphicx}
\usepackage{ascmac}
\alignbeforeskip -5mm
\alignafterskip -5mm
\eqnarraybeforeskip -5mm
\eqnarrayafterskip -5mm

\makeatletter
\newenvironment{figurehere}
  {\def\@captype{figure}}
  {}
\makeatother

\jptitle{ビーバーコンテストタイトル\\フランスのソース}
\etitle{}
\jpauthor{鈴木一至 佐々木陽広}
\eauthor{Kazushi Suzuki,Akihiro Sasaki}
\course{谷聖一研究室}
\year{27}


%% 概要 %%
\abstract{

コンピュータ・サイエンスの普及を目的とした取り組みは,様々なところで行われている.
その中の1つに,小・中・高校生を対象にした「ビーバーコンテスト」がある.本演習では、フランスでオープンソースとして公開されているビーバーコンテストのサーバーを元に、実際のビーバーコンテストで行われた過去のコンテストを受けられるWebサイトを実装した。}
%\compheading

\begin{document}
\maketitle
\begin{multicols}{2}
\setcounter{page}{1}

\section{はじめに}

\subsection{ビーバーコンテストとは}
ビーバーコンテスト(\cite{bebras-contest, bebras-pdf})とは,2004年にリトアニアで始められた国際的な情報科学コンテストである.日本の小学5年生から高等学校3年生を対象とし、年に1回開催される.日本は2011年度から正式参加しており,2014年には世界35カ国から92万人の児童・生徒が参加している.情報科学の事前知識がなくても解くことが可能な問題を扱い,問題に取り組むことによって情報科学の基礎概念に触れることができ,論理的思考の向上の一助になるようなものになっている.また,コンテスト後に参加者同士で問題の内容について議論することで,情報科学に興味を持つきっかけになることが期待される.

\subsection{目的}
現在、ビーバーコンテストを受けるに離れたオランダのサーバーに接続する必要があり、コンテストを受ける環境によっては上手く問題を受けることができない場合があった.この問題を解消するために、手元の環境で実装し、主に日本から過去のコンテストを受ける際に正常に動作できるようにすることが本演習の目的である.
また、今後行われるコンテストの

\section{実装内容}
非対話型問題のうち,4題が,試行錯誤させることで児童・生徒の理解促進がはかれると考え,試行錯誤可能なコンテンツを試作した.以下で試作したコンテンツを説明する.

\subsection{実装項目1}
実装項目1を実行した\\a\\a\\a\\a\\a


\section{終わりに}
a\\おわりにまとめ
\\a
\\a
\\a

\end{multicols}

%%%%% 参考文献 %%%%%
\begin{thebibliography}{1}

\bibitem{bebras-contest} 「ビーバーコンテスト」情報ページ.  http://bebras.eplang.jp/ , (参照 2014-01-16)
\bibitem{bebras-pdf} 谷 聖一, 兼宗 進, 井戸坂 幸男. 小中高生向け国際情報科学コンテストBebras.  http://www.ipsj.or.jp/magazine/9faeag0000005al5-att/peta55-11.pdf, (参照 2015-01-23)



\end{thebibliography}

\end{document} 




